\section{Introduction}

\subsection{Theory}
An equation known as Stokes' Law expresses the drag force that a spherical object encounters when travelling through a fluid at a constant speed. The drag force is influenced by a number of variables, including the size of the object and the fluid's viscosity.

A fluid's viscosity is a gauge of how difficult it is for it to flow. Fluids with greater viscosities tend to be thicker and more flow-resistive, whereas fluids with lower viscosities tend to be thinner and less flow-resistive.
By measuring the drag force a sphere experiences as it moves through a fluid at a constant speed, Stokes' Law offers a way for calculating a fluid's viscosity. The equation links the gravitational constant, the fluid's viscosity, the object's velocity, and its radius to the drag force that an object experiences.

Stokes' Law is frequently utilized in a wide range of scientific and technical applications, including the analysis of sedimentation rates in geological processes, the research of fluid dynamics, and the design of pumps and turbines. By accurately measuring the viscosity of a fluid using Stokes' Law, scientists and engineers can gain valuable insights into the behavior of fluids in various contexts, leading to improved designs and processes in a wide range of industries.



\subsection{Equipment}
The equipment used in this experiment included:
\begin{itemize}
	\item Cylindrical tank with examined fluid
\item Aerometer
\item Set of balls
\item  Scales
\item Micrometric screw
\item Ruler with millimeter scale
\item Stopwatch

\end{itemize}


