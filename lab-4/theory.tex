\section{Introduction}

\subsection{Theory}
The measurement of the speed of light is a fundamental experiment in the field of physics, as the speed of light is one of the fundamental constants of the universe. The speed of light in a vacuum, denoted by the symbol c, is approximately 299,792,458 meters per second, and is a fundamental constant that underpins many theories in physics. The speed of light plays a crucial role in the fields of astronomy, cosmology, and quantum mechanics, among others.

There are many methods to measure the speed of light, and one of the earliest and most famous was performed by the Danish astronomer Ole R�mer in the late 17th century. R�mer observed the moons of Jupiter and noted that their observed periods varied as the distance between Jupiter and Earth changed due to the Earth's orbit around the Sun. By measuring the time delay between different positions of the Earth, R�mer was able to estimate the speed of light.

\subsection{Equipment}
The equipment used in this experiment included:
\begin{itemize}
	\item Device for measuring the speed of light with a phase shifter and transmitting and receiving diodes
	\item Oscilloscope
	\item Mirror system
	\item Converging lens
	\item Ruler with a scale
	\item Liquid cells.
\end{itemize}


