\section{Introduction}

\subsection{Theory}
The measurement of the speed of light is a fundamental experiment in the field of physics, as the speed of light is one of the fundamental constants of the universe. The speed of light in a vacuum, denoted by the symbol c, is approximately 299,792,458 meters per second, and is a fundamental constant that underpins many theories in physics. 

Speed of light plays a significant role in helping us understand the universe and its workings. The velocity of light in various materials, such as liquids and air, is essential for analyzing how it moves through these substances. Such analysis can aid in understanding the refractive index, a crucial aspect of optics research. This paper introduces an innovative and straightforward technique, known as the folding method, to measure the speed of light accurately. We use this method to determine the speed of light in liquids and air, which, in turn, helps us calculate their respective refractive indices. By presenting this information, we aim to demonstrate the effectiveness and simplicity of the folding method in enhancing our knowledge of how light behaves in different substances, which can have implications for optical and other scientific research areas.


\subsection{Equipment}
The equipment used in this experiment included:
\begin{itemize}
	\item Device for measuring the speed of light with a phase shifter and transmitting and receiving diodes
	\item Oscilloscope
	\item Mirror system
	\item Converging lens
	\item Ruler with a scale
	\item Liquid cells.
\end{itemize}


