\section{Introduction}

\subsection{Theory}
A well-known optical experiment uses Newton's rings to determine the wavelength of light and the radius of curvature of a lens. When a convex lens is put on a flat glass surface and illuminated by a monochromatic light source, Newton's rings are created. Concentric bright and dark circles with bright rings denoting constructive interference and dark rings denoting destructive interference make up the interference pattern created by the reflected light. The wavelength of the light can also be estimated by knowing the refractive index of the medium between the lens and the flat glass surface as well as the radius of curvature of the lens. This experiment is widely used in the field of optics and provides a practical demonstration of interference phenomena and its applications in optics.


\subsection{Equipment}
The equipment used in this experiment included:
\begin{itemize}
	\item Microscope with compound table and mounted lens
	\item Sodium-vapour lamp with power supply

\end{itemize}


