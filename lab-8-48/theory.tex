\section{Introduction}

\subsection{Theory}
The study of quantum mechanics has revolutionized our understanding of the behavior of particles and electromagnetic radiation at the atomic and subatomic levels. One of the fundamental constants in quantum mechanics is Planck's constant (h), which relates the energy of a photon to its frequency. The determination of Planck's constant is of significant importance in various fields, including physics, electronics, and materials science.

Light-emitting diodes (LEDs) provide a practical and accessible platform for investigating the relationship between energy and frequency through their current-voltage characteristics. When a forward voltage is applied to an LED, it emits light with a characteristic wavelength. By examining the current-voltage behavior of different LEDs, we can obtain valuable insights into the energy levels involved in the photon emission process.
\subsection{Equipment}

The following tools were used during the laboratory:

\begin{itemize}
\item Tunable power supply 
\item Electroluminescent diode
\item Digital multimeters
\item Monochromator
\item Photoresistor
\end{itemize}

