\section{Introduction}

\subsection{Theory}
 
The study of electrical circuits and their behavior is a fundamental aspect of physics. In this experiment, we aim to experimentally verify Ohm's Law for alternating current and determine the inductance and capacitance of selected components. 

To conduct this experiment, we will utilize a power generator OWON AG 1022F, a voltmeter, an ammeter, and a series RLC circuit consisting of resistors (R), capacitors (C), and inductors (L) assembly, and wires. 

The goals of this experiment are three-fold: first, to experimentally verify Ohm's Law for alternating current; second, to determine the inductance L of the selected inductor component; and third, to determine the capacitance C of the selected capacitor component. 

To achieve these goals, we will set up the experimental apparatus as per Figure 1 and connect the electric circuit for the RC, RL, and RLC circuits as per Figures 2, 3, and 4, respectively. The RC circuit contains only R and C components connected in series, the RL circuit contains only R and L components connected in series, and the RLC circuit contains R, L, and C components connected in series. 

For the RC and RL circuits, we will measure the current I and voltage U dependencies for the selected frequency f. For the RLC circuit, we will measure the current I and voltage U dependencies for the same selected frequency f and determine the impedance of the circuit. 

The experimental data obtained will be used to calculate the capacitance C of the capacitor component for the RC circuit, the inductance L of the inductor component 

\subsection{Equipment}
The equipment used in this experiment included
\begin{itemize}
	\item power generator
	\item voltmeter
	\item ammeter
	\item RLC assembly
\end{itemize}
