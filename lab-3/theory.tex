\section{Introduction}

\subsection{Theory}
 Current saturation - it is what we call when no more current can be produced by a device(semiconductor or magnetic) if we vary the voltage or amount of light shun onto it. A physical reason will stop this from happening, and it varies on circumstance as to why this might happen.
 
 The external photoelectric effect is a phenomenon in which electrons are ejected from the surface of a material when light of sufficient energy is incident upon it. The effect was first observed by Heinrich Hertz in 1887, and it is a key concept in the understanding of the interaction between light and matter.

The theory of the external photoelectric effect is based on the idea that light consists of packets of energy called photons. When a photon of sufficient energy strikes an atom or molecule in a material, it can transfer its energy to an electron in the material, causing the electron to be ejected from the surface. The minimum energy required for this to occur is known as the material's work function, which is the energy required to remove an electron from the material's surface.

The external photoelectric effect is described by the photoelectric equation, which relates the energy of the incident photons, the work function of the material, and the maximum kinetic energy of the ejected electrons. The equation is given by:

$E_{\text{photon}} = \phi + \frac{1}{2}mv_{\text{max}}^2$

where $E_{\text{photon}}$ is the energy of the incident photon, $\phi$ is the work function of the material, $m$ is the mass of the ejected electron, and $v_{\text{max}}$ is the maximum velocity (and thus kinetic energy) of the ejected electron.

The photoelectric effect is an important phenomenon in many areas of science and technology, including solar energy conversion, semiconductor electronics, and imaging technologies. It has also played a key role in the development of quantum mechanics, as it provided early evidence for the wave-particle duality of light.

\subsection{Equipment}
The equipment used in this experiment included:
\begin{itemize}
	\item Monochromator
	\item DC ammeter Sanwa CD771
	\item DC voltmeter Sanwa CD771
	\item Illuminator
	\item DC power supply for the illuminator
	\item Photocell power supply
\end{itemize}


